\section{Introduction}\label{sec:intro}
LD\_PRELOAD is an environment variable in linux. 
It can be used to load a customized shared library at runtime and before all other, standard objects.
LD\_PRELOAD can therefore be used to change the behavior of system calls. 
This does not manipulate the syscalls directly, but the glibc wrappers for them. 
The modified functions are only called if they have been linked dynamically. With static linking, LD\_PRELOAD has no effect. 
In this project, the following seven functions are modified: getchar(), read(), open(), write(), execve(), connect() and malloc().

\subsection{Subsection Figures} %% t=top, b=bottom, h=here


\subsection{Subsection Tables}

\section{Methology}
The implementation consists of seven hijacked functions that have been combined into a library at the end.
When activated in a terminal session with "export LD\_PRELOAD=./hacked.so", they give the user the impression that his computer has been seriously hacked.

\subsection{Open()}
The hijacked open() function has three main elements:
 when a user tries to open the file "secrets.txt", the function creates a new txt file with the content "you should not be so curious".
If the user now writes something, it will be written into this new file, so the user has no way of reading or modifying the secrets. 
It is also not possible to copy the file to a new one. If the user tries to open the file openThis.txt, he will get a message that he has been hacked and should not do everything he is told. 
The last manipulation concerns the preloadLib file. This is where the code from the hijacks is implemented, so the user should not be able to manipulate it.

\subsection{Write()}
The Write function manipulates the terminal output for an invalid command. 
If the user types a command such as “invalidcommand”, it is normally displayed that this command was not found. 
With the new function, the user is misled by replacing the output with “finished execution: no errors”, so that he does not realize that his command has done nothing.

\subsection{Read()}
The manipulated read function affects the commands cat, bash and less. 
The commands are not executed, instead the terminal displays in red that unauthorized access has been detected. 
Then various directories are listed as deleted to give the impression that the user is losing all his files. 
The control + C command is deactivated in the meantime, simulating total loss of control.

\subsection{Connect()}
The connect() system call establishes a connection between the socket referred to by the file descriptor sockfd and the address specified by addr. The hijacked connect function blocks all incoming and outgoing internet connections so the user can’t connect to the internet via the terminal using terminal web browsers such as w3m or curl. The hijacked function also has the task of blocking all connections to ports 6667, 6697, 993 which are responsible for internet relay chat(IRC), IRC over TLS/SSL and secure internet message access protocol(IMAP) used for email communication. Therefore, when the hijacked file is exported with LD_PRELOAD on the terminal, connecting to the internet will be impossible through non-sandboxed and low security apps like geary, liferea, hexchat and gnome-weather. 
Bash script: To make this hijacking more permanent, we have played with the bashrc file by writing in it the export LD_PRELOAD so that every terminal opened while that line is in the bashrc file, would induce the hijacking, making it more difficult to get rid of. In this bash script we have also added functionalities for renaming the culprit file downloaded, replacing it somewhere random within the user’s directories and also deleting the source.so file so that the detection is extremely difficult. This way the user will not know what has happened to them and would have to take a lot of time to find the cause. The only way to revert the permanent changes done to every terminal would be to unset LD_PRELOAD at the beginning of every terminal or to delete the injected export LD_PRELOAD lines from the bashrc script.


\subsection{Lessons learned}
We all learned a lot from this project.
 Not only in terms of programming in C in the Linux environment, which was still quite foreign to us, but also in terms of teamwork. 
 Our project was divided up well among the team members according to the individual functions. This was particularly helpful at the beginning, so that we could all work at our own pace, and showed us how important it is in projects to divide the work evenly and clearly from the start. \\
 We looked in detail at the behavior of the individual functions, which gave us a valuable insight into the commands used on a daily basis. \\
 We also found out while putting the functions together that most of these functions work simultaneously and are called at the same time when something is executed. Therefore putting them together and still keeping their individual functionalities was very challenging because we had to find a way for them to go around each other. This is where we thought of wrapper variables that can activate or deactivate the hijack function either equaling to 1 or 0. To demonstrate, geary is the most securely programmed app that we use in this project, and when geary is run in the command line, to open it calls the system call function write(). To avoid this and still have geary working, we had to put a variable wrapper around write() so that when its set to 1, the write() hijack would not work anymore, letting us open geary from the terminal. This was also done with the gethchar() function.
